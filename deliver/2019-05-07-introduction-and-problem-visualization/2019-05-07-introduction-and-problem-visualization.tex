
% Default to the notebook output style

    


% Inherit from the specified cell style.




    
\documentclass[11pt]{article}

    
    
    \usepackage[T1]{fontenc}
    % Nicer default font (+ math font) than Computer Modern for most use cases
    \usepackage{mathpazo}

    % Basic figure setup, for now with no caption control since it's done
    % automatically by Pandoc (which extracts ![](path) syntax from Markdown).
    \usepackage{graphicx}
    % We will generate all images so they have a width \maxwidth. This means
    % that they will get their normal width if they fit onto the page, but
    % are scaled down if they would overflow the margins.
    \makeatletter
    \def\maxwidth{\ifdim\Gin@nat@width>\linewidth\linewidth
    \else\Gin@nat@width\fi}
    \makeatother
    \let\Oldincludegraphics\includegraphics
    % Set max figure width to be 80% of text width, for now hardcoded.
    \renewcommand{\includegraphics}[1]{\Oldincludegraphics[width=.8\maxwidth]{#1}}
    % Ensure that by default, figures have no caption (until we provide a
    % proper Figure object with a Caption API and a way to capture that
    % in the conversion process - todo).
    \usepackage{caption}
    \DeclareCaptionLabelFormat{nolabel}{}
    \captionsetup{labelformat=nolabel}

    \usepackage{adjustbox} % Used to constrain images to a maximum size 
    \usepackage{xcolor} % Allow colors to be defined
    \usepackage{enumerate} % Needed for markdown enumerations to work
    \usepackage{geometry} % Used to adjust the document margins
    \usepackage{amsmath} % Equations
    \usepackage{amssymb} % Equations
    \usepackage{textcomp} % defines textquotesingle
    % Hack from http://tex.stackexchange.com/a/47451/13684:
    \AtBeginDocument{%
        \def\PYZsq{\textquotesingle}% Upright quotes in Pygmentized code
    }
    \usepackage{upquote} % Upright quotes for verbatim code
    \usepackage{eurosym} % defines \euro
    \usepackage[mathletters]{ucs} % Extended unicode (utf-8) support
    \usepackage[utf8x]{inputenc} % Allow utf-8 characters in the tex document
    \usepackage{fancyvrb} % verbatim replacement that allows latex
    \usepackage{grffile} % extends the file name processing of package graphics 
                         % to support a larger range 
    % The hyperref package gives us a pdf with properly built
    % internal navigation ('pdf bookmarks' for the table of contents,
    % internal cross-reference links, web links for URLs, etc.)
    \usepackage{hyperref}
    \usepackage{longtable} % longtable support required by pandoc >1.10
    \usepackage{booktabs}  % table support for pandoc > 1.12.2
    \usepackage[inline]{enumitem} % IRkernel/repr support (it uses the enumerate* environment)
    \usepackage[normalem]{ulem} % ulem is needed to support strikethroughs (\sout)
                                % normalem makes italics be italics, not underlines
    \usepackage{mathrsfs}
    

    
    
    % Colors for the hyperref package
    \definecolor{urlcolor}{rgb}{0,.145,.698}
    \definecolor{linkcolor}{rgb}{.71,0.21,0.01}
    \definecolor{citecolor}{rgb}{.12,.54,.11}

    % ANSI colors
    \definecolor{ansi-black}{HTML}{3E424D}
    \definecolor{ansi-black-intense}{HTML}{282C36}
    \definecolor{ansi-red}{HTML}{E75C58}
    \definecolor{ansi-red-intense}{HTML}{B22B31}
    \definecolor{ansi-green}{HTML}{00A250}
    \definecolor{ansi-green-intense}{HTML}{007427}
    \definecolor{ansi-yellow}{HTML}{DDB62B}
    \definecolor{ansi-yellow-intense}{HTML}{B27D12}
    \definecolor{ansi-blue}{HTML}{208FFB}
    \definecolor{ansi-blue-intense}{HTML}{0065CA}
    \definecolor{ansi-magenta}{HTML}{D160C4}
    \definecolor{ansi-magenta-intense}{HTML}{A03196}
    \definecolor{ansi-cyan}{HTML}{60C6C8}
    \definecolor{ansi-cyan-intense}{HTML}{258F8F}
    \definecolor{ansi-white}{HTML}{C5C1B4}
    \definecolor{ansi-white-intense}{HTML}{A1A6B2}
    \definecolor{ansi-default-inverse-fg}{HTML}{FFFFFF}
    \definecolor{ansi-default-inverse-bg}{HTML}{000000}

    % commands and environments needed by pandoc snippets
    % extracted from the output of `pandoc -s`
    \providecommand{\tightlist}{%
      \setlength{\itemsep}{0pt}\setlength{\parskip}{0pt}}
    \DefineVerbatimEnvironment{Highlighting}{Verbatim}{commandchars=\\\{\}}
    % Add ',fontsize=\small' for more characters per line
    \newenvironment{Shaded}{}{}
    \newcommand{\KeywordTok}[1]{\textcolor[rgb]{0.00,0.44,0.13}{\textbf{{#1}}}}
    \newcommand{\DataTypeTok}[1]{\textcolor[rgb]{0.56,0.13,0.00}{{#1}}}
    \newcommand{\DecValTok}[1]{\textcolor[rgb]{0.25,0.63,0.44}{{#1}}}
    \newcommand{\BaseNTok}[1]{\textcolor[rgb]{0.25,0.63,0.44}{{#1}}}
    \newcommand{\FloatTok}[1]{\textcolor[rgb]{0.25,0.63,0.44}{{#1}}}
    \newcommand{\CharTok}[1]{\textcolor[rgb]{0.25,0.44,0.63}{{#1}}}
    \newcommand{\StringTok}[1]{\textcolor[rgb]{0.25,0.44,0.63}{{#1}}}
    \newcommand{\CommentTok}[1]{\textcolor[rgb]{0.38,0.63,0.69}{\textit{{#1}}}}
    \newcommand{\OtherTok}[1]{\textcolor[rgb]{0.00,0.44,0.13}{{#1}}}
    \newcommand{\AlertTok}[1]{\textcolor[rgb]{1.00,0.00,0.00}{\textbf{{#1}}}}
    \newcommand{\FunctionTok}[1]{\textcolor[rgb]{0.02,0.16,0.49}{{#1}}}
    \newcommand{\RegionMarkerTok}[1]{{#1}}
    \newcommand{\ErrorTok}[1]{\textcolor[rgb]{1.00,0.00,0.00}{\textbf{{#1}}}}
    \newcommand{\NormalTok}[1]{{#1}}
    
    % Additional commands for more recent versions of Pandoc
    \newcommand{\ConstantTok}[1]{\textcolor[rgb]{0.53,0.00,0.00}{{#1}}}
    \newcommand{\SpecialCharTok}[1]{\textcolor[rgb]{0.25,0.44,0.63}{{#1}}}
    \newcommand{\VerbatimStringTok}[1]{\textcolor[rgb]{0.25,0.44,0.63}{{#1}}}
    \newcommand{\SpecialStringTok}[1]{\textcolor[rgb]{0.73,0.40,0.53}{{#1}}}
    \newcommand{\ImportTok}[1]{{#1}}
    \newcommand{\DocumentationTok}[1]{\textcolor[rgb]{0.73,0.13,0.13}{\textit{{#1}}}}
    \newcommand{\AnnotationTok}[1]{\textcolor[rgb]{0.38,0.63,0.69}{\textbf{\textit{{#1}}}}}
    \newcommand{\CommentVarTok}[1]{\textcolor[rgb]{0.38,0.63,0.69}{\textbf{\textit{{#1}}}}}
    \newcommand{\VariableTok}[1]{\textcolor[rgb]{0.10,0.09,0.49}{{#1}}}
    \newcommand{\ControlFlowTok}[1]{\textcolor[rgb]{0.00,0.44,0.13}{\textbf{{#1}}}}
    \newcommand{\OperatorTok}[1]{\textcolor[rgb]{0.40,0.40,0.40}{{#1}}}
    \newcommand{\BuiltInTok}[1]{{#1}}
    \newcommand{\ExtensionTok}[1]{{#1}}
    \newcommand{\PreprocessorTok}[1]{\textcolor[rgb]{0.74,0.48,0.00}{{#1}}}
    \newcommand{\AttributeTok}[1]{\textcolor[rgb]{0.49,0.56,0.16}{{#1}}}
    \newcommand{\InformationTok}[1]{\textcolor[rgb]{0.38,0.63,0.69}{\textbf{\textit{{#1}}}}}
    \newcommand{\WarningTok}[1]{\textcolor[rgb]{0.38,0.63,0.69}{\textbf{\textit{{#1}}}}}
    
    
    % Define a nice break command that doesn't care if a line doesn't already
    % exist.
    \def\br{\hspace*{\fill} \\* }
    % Math Jax compatibility definitions
    \def\gt{>}
    \def\lt{<}
    \let\Oldtex\TeX
    \let\Oldlatex\LaTeX
    \renewcommand{\TeX}{\textrm{\Oldtex}}
    \renewcommand{\LaTeX}{\textrm{\Oldlatex}}
    % Document parameters
    % Document title
    \title{2019-05-07-introduction-and-problem-visualization}
    
    
    
    
    

    % Pygments definitions
    
\makeatletter
\def\PY@reset{\let\PY@it=\relax \let\PY@bf=\relax%
    \let\PY@ul=\relax \let\PY@tc=\relax%
    \let\PY@bc=\relax \let\PY@ff=\relax}
\def\PY@tok#1{\csname PY@tok@#1\endcsname}
\def\PY@toks#1+{\ifx\relax#1\empty\else%
    \PY@tok{#1}\expandafter\PY@toks\fi}
\def\PY@do#1{\PY@bc{\PY@tc{\PY@ul{%
    \PY@it{\PY@bf{\PY@ff{#1}}}}}}}
\def\PY#1#2{\PY@reset\PY@toks#1+\relax+\PY@do{#2}}

\expandafter\def\csname PY@tok@gd\endcsname{\def\PY@tc##1{\textcolor[rgb]{0.63,0.00,0.00}{##1}}}
\expandafter\def\csname PY@tok@gu\endcsname{\let\PY@bf=\textbf\def\PY@tc##1{\textcolor[rgb]{0.50,0.00,0.50}{##1}}}
\expandafter\def\csname PY@tok@gt\endcsname{\def\PY@tc##1{\textcolor[rgb]{0.00,0.27,0.87}{##1}}}
\expandafter\def\csname PY@tok@gs\endcsname{\let\PY@bf=\textbf}
\expandafter\def\csname PY@tok@gr\endcsname{\def\PY@tc##1{\textcolor[rgb]{1.00,0.00,0.00}{##1}}}
\expandafter\def\csname PY@tok@cm\endcsname{\let\PY@it=\textit\def\PY@tc##1{\textcolor[rgb]{0.25,0.50,0.50}{##1}}}
\expandafter\def\csname PY@tok@vg\endcsname{\def\PY@tc##1{\textcolor[rgb]{0.10,0.09,0.49}{##1}}}
\expandafter\def\csname PY@tok@vi\endcsname{\def\PY@tc##1{\textcolor[rgb]{0.10,0.09,0.49}{##1}}}
\expandafter\def\csname PY@tok@vm\endcsname{\def\PY@tc##1{\textcolor[rgb]{0.10,0.09,0.49}{##1}}}
\expandafter\def\csname PY@tok@mh\endcsname{\def\PY@tc##1{\textcolor[rgb]{0.40,0.40,0.40}{##1}}}
\expandafter\def\csname PY@tok@cs\endcsname{\let\PY@it=\textit\def\PY@tc##1{\textcolor[rgb]{0.25,0.50,0.50}{##1}}}
\expandafter\def\csname PY@tok@ge\endcsname{\let\PY@it=\textit}
\expandafter\def\csname PY@tok@vc\endcsname{\def\PY@tc##1{\textcolor[rgb]{0.10,0.09,0.49}{##1}}}
\expandafter\def\csname PY@tok@il\endcsname{\def\PY@tc##1{\textcolor[rgb]{0.40,0.40,0.40}{##1}}}
\expandafter\def\csname PY@tok@go\endcsname{\def\PY@tc##1{\textcolor[rgb]{0.53,0.53,0.53}{##1}}}
\expandafter\def\csname PY@tok@cp\endcsname{\def\PY@tc##1{\textcolor[rgb]{0.74,0.48,0.00}{##1}}}
\expandafter\def\csname PY@tok@gi\endcsname{\def\PY@tc##1{\textcolor[rgb]{0.00,0.63,0.00}{##1}}}
\expandafter\def\csname PY@tok@gh\endcsname{\let\PY@bf=\textbf\def\PY@tc##1{\textcolor[rgb]{0.00,0.00,0.50}{##1}}}
\expandafter\def\csname PY@tok@ni\endcsname{\let\PY@bf=\textbf\def\PY@tc##1{\textcolor[rgb]{0.60,0.60,0.60}{##1}}}
\expandafter\def\csname PY@tok@nl\endcsname{\def\PY@tc##1{\textcolor[rgb]{0.63,0.63,0.00}{##1}}}
\expandafter\def\csname PY@tok@nn\endcsname{\let\PY@bf=\textbf\def\PY@tc##1{\textcolor[rgb]{0.00,0.00,1.00}{##1}}}
\expandafter\def\csname PY@tok@no\endcsname{\def\PY@tc##1{\textcolor[rgb]{0.53,0.00,0.00}{##1}}}
\expandafter\def\csname PY@tok@na\endcsname{\def\PY@tc##1{\textcolor[rgb]{0.49,0.56,0.16}{##1}}}
\expandafter\def\csname PY@tok@nb\endcsname{\def\PY@tc##1{\textcolor[rgb]{0.00,0.50,0.00}{##1}}}
\expandafter\def\csname PY@tok@nc\endcsname{\let\PY@bf=\textbf\def\PY@tc##1{\textcolor[rgb]{0.00,0.00,1.00}{##1}}}
\expandafter\def\csname PY@tok@nd\endcsname{\def\PY@tc##1{\textcolor[rgb]{0.67,0.13,1.00}{##1}}}
\expandafter\def\csname PY@tok@ne\endcsname{\let\PY@bf=\textbf\def\PY@tc##1{\textcolor[rgb]{0.82,0.25,0.23}{##1}}}
\expandafter\def\csname PY@tok@nf\endcsname{\def\PY@tc##1{\textcolor[rgb]{0.00,0.00,1.00}{##1}}}
\expandafter\def\csname PY@tok@si\endcsname{\let\PY@bf=\textbf\def\PY@tc##1{\textcolor[rgb]{0.73,0.40,0.53}{##1}}}
\expandafter\def\csname PY@tok@s2\endcsname{\def\PY@tc##1{\textcolor[rgb]{0.73,0.13,0.13}{##1}}}
\expandafter\def\csname PY@tok@nt\endcsname{\let\PY@bf=\textbf\def\PY@tc##1{\textcolor[rgb]{0.00,0.50,0.00}{##1}}}
\expandafter\def\csname PY@tok@nv\endcsname{\def\PY@tc##1{\textcolor[rgb]{0.10,0.09,0.49}{##1}}}
\expandafter\def\csname PY@tok@s1\endcsname{\def\PY@tc##1{\textcolor[rgb]{0.73,0.13,0.13}{##1}}}
\expandafter\def\csname PY@tok@dl\endcsname{\def\PY@tc##1{\textcolor[rgb]{0.73,0.13,0.13}{##1}}}
\expandafter\def\csname PY@tok@ch\endcsname{\let\PY@it=\textit\def\PY@tc##1{\textcolor[rgb]{0.25,0.50,0.50}{##1}}}
\expandafter\def\csname PY@tok@m\endcsname{\def\PY@tc##1{\textcolor[rgb]{0.40,0.40,0.40}{##1}}}
\expandafter\def\csname PY@tok@gp\endcsname{\let\PY@bf=\textbf\def\PY@tc##1{\textcolor[rgb]{0.00,0.00,0.50}{##1}}}
\expandafter\def\csname PY@tok@sh\endcsname{\def\PY@tc##1{\textcolor[rgb]{0.73,0.13,0.13}{##1}}}
\expandafter\def\csname PY@tok@ow\endcsname{\let\PY@bf=\textbf\def\PY@tc##1{\textcolor[rgb]{0.67,0.13,1.00}{##1}}}
\expandafter\def\csname PY@tok@sx\endcsname{\def\PY@tc##1{\textcolor[rgb]{0.00,0.50,0.00}{##1}}}
\expandafter\def\csname PY@tok@bp\endcsname{\def\PY@tc##1{\textcolor[rgb]{0.00,0.50,0.00}{##1}}}
\expandafter\def\csname PY@tok@c1\endcsname{\let\PY@it=\textit\def\PY@tc##1{\textcolor[rgb]{0.25,0.50,0.50}{##1}}}
\expandafter\def\csname PY@tok@fm\endcsname{\def\PY@tc##1{\textcolor[rgb]{0.00,0.00,1.00}{##1}}}
\expandafter\def\csname PY@tok@o\endcsname{\def\PY@tc##1{\textcolor[rgb]{0.40,0.40,0.40}{##1}}}
\expandafter\def\csname PY@tok@kc\endcsname{\let\PY@bf=\textbf\def\PY@tc##1{\textcolor[rgb]{0.00,0.50,0.00}{##1}}}
\expandafter\def\csname PY@tok@c\endcsname{\let\PY@it=\textit\def\PY@tc##1{\textcolor[rgb]{0.25,0.50,0.50}{##1}}}
\expandafter\def\csname PY@tok@mf\endcsname{\def\PY@tc##1{\textcolor[rgb]{0.40,0.40,0.40}{##1}}}
\expandafter\def\csname PY@tok@err\endcsname{\def\PY@bc##1{\setlength{\fboxsep}{0pt}\fcolorbox[rgb]{1.00,0.00,0.00}{1,1,1}{\strut ##1}}}
\expandafter\def\csname PY@tok@mb\endcsname{\def\PY@tc##1{\textcolor[rgb]{0.40,0.40,0.40}{##1}}}
\expandafter\def\csname PY@tok@ss\endcsname{\def\PY@tc##1{\textcolor[rgb]{0.10,0.09,0.49}{##1}}}
\expandafter\def\csname PY@tok@sr\endcsname{\def\PY@tc##1{\textcolor[rgb]{0.73,0.40,0.53}{##1}}}
\expandafter\def\csname PY@tok@mo\endcsname{\def\PY@tc##1{\textcolor[rgb]{0.40,0.40,0.40}{##1}}}
\expandafter\def\csname PY@tok@kd\endcsname{\let\PY@bf=\textbf\def\PY@tc##1{\textcolor[rgb]{0.00,0.50,0.00}{##1}}}
\expandafter\def\csname PY@tok@mi\endcsname{\def\PY@tc##1{\textcolor[rgb]{0.40,0.40,0.40}{##1}}}
\expandafter\def\csname PY@tok@kn\endcsname{\let\PY@bf=\textbf\def\PY@tc##1{\textcolor[rgb]{0.00,0.50,0.00}{##1}}}
\expandafter\def\csname PY@tok@cpf\endcsname{\let\PY@it=\textit\def\PY@tc##1{\textcolor[rgb]{0.25,0.50,0.50}{##1}}}
\expandafter\def\csname PY@tok@kr\endcsname{\let\PY@bf=\textbf\def\PY@tc##1{\textcolor[rgb]{0.00,0.50,0.00}{##1}}}
\expandafter\def\csname PY@tok@s\endcsname{\def\PY@tc##1{\textcolor[rgb]{0.73,0.13,0.13}{##1}}}
\expandafter\def\csname PY@tok@kp\endcsname{\def\PY@tc##1{\textcolor[rgb]{0.00,0.50,0.00}{##1}}}
\expandafter\def\csname PY@tok@w\endcsname{\def\PY@tc##1{\textcolor[rgb]{0.73,0.73,0.73}{##1}}}
\expandafter\def\csname PY@tok@kt\endcsname{\def\PY@tc##1{\textcolor[rgb]{0.69,0.00,0.25}{##1}}}
\expandafter\def\csname PY@tok@sc\endcsname{\def\PY@tc##1{\textcolor[rgb]{0.73,0.13,0.13}{##1}}}
\expandafter\def\csname PY@tok@sb\endcsname{\def\PY@tc##1{\textcolor[rgb]{0.73,0.13,0.13}{##1}}}
\expandafter\def\csname PY@tok@sa\endcsname{\def\PY@tc##1{\textcolor[rgb]{0.73,0.13,0.13}{##1}}}
\expandafter\def\csname PY@tok@k\endcsname{\let\PY@bf=\textbf\def\PY@tc##1{\textcolor[rgb]{0.00,0.50,0.00}{##1}}}
\expandafter\def\csname PY@tok@se\endcsname{\let\PY@bf=\textbf\def\PY@tc##1{\textcolor[rgb]{0.73,0.40,0.13}{##1}}}
\expandafter\def\csname PY@tok@sd\endcsname{\let\PY@it=\textit\def\PY@tc##1{\textcolor[rgb]{0.73,0.13,0.13}{##1}}}

\def\PYZbs{\char`\\}
\def\PYZus{\char`\_}
\def\PYZob{\char`\{}
\def\PYZcb{\char`\}}
\def\PYZca{\char`\^}
\def\PYZam{\char`\&}
\def\PYZlt{\char`\<}
\def\PYZgt{\char`\>}
\def\PYZsh{\char`\#}
\def\PYZpc{\char`\%}
\def\PYZdl{\char`\$}
\def\PYZhy{\char`\-}
\def\PYZsq{\char`\'}
\def\PYZdq{\char`\"}
\def\PYZti{\char`\~}
% for compatibility with earlier versions
\def\PYZat{@}
\def\PYZlb{[}
\def\PYZrb{]}
\makeatother


    % Exact colors from NB
    \definecolor{incolor}{rgb}{0.0, 0.0, 0.5}
    \definecolor{outcolor}{rgb}{0.545, 0.0, 0.0}



    
    % Prevent overflowing lines due to hard-to-break entities
    \sloppy 
    % Setup hyperref package
    \hypersetup{
      breaklinks=true,  % so long urls are correctly broken across lines
      colorlinks=true,
      urlcolor=urlcolor,
      linkcolor=linkcolor,
      citecolor=citecolor,
      }
    % Slightly bigger margins than the latex defaults
    
    \geometry{verbose,tmargin=1in,bmargin=1in,lmargin=1in,rmargin=1in}
    
    

    \begin{document}
    
    
    \maketitle
    
    

    
    \section{Introdução}\label{introduuxe7uxe3o}

Problemas envolvendo a otimização simultânea de múltiplos objetivos
ganham cada vez mais evidência com o avanço tecnológico. Há um aumento
de interesse por formulações matemáticas voltadas para problemas de
otimização multiobjetivo e de tomada de decisão multicritério, os quais
possuem frentes de pesquisa bem consolidadas e com as seguintes
estratégias:

\begin{enumerate}
\def\labelenumi{\arabic{enumi}.}
\item
  Otimização multiobjetivo (do inglês \emph{Multi-Objective
  Optimization} -- MOO) (Branke et al., 2008): busca amostrar a
  fronteira de Pareto, composta por soluções eficientes do problema e
  que apresentam diferentes compromissos entre os objetivos, permitindo
  que a preferência do tomador de decisão seja definida e aplicada a
  posteriori.
\item
  Tomada de decisão multicritério (do inglês \emph{Multicriteria
  Decision Making} -- MCDM) (Köksalan et al., 2011; Steuer, 1986): busca
  explorar a preferência a priori do tomador de decisão, de modo a
  ordenar por mérito as múltiplas alternativas de soluções eficientes
  existentes.
\end{enumerate}

Dentre as técnicas já propostas para resolver problemas MOO, destacam-se
as metaheurísticas de otimização, em especial as que recorrem a
estratégias de busca populacional (Coello Coello et al., 2007). Essas
metaheurísticas centram seus esforços na descoberta de novas soluções
candidatas que sejam diversas entre si e não-dominadas pelas demais
soluções candidatas já des- cobertas, num processo iterativo. Nesta
busca iterativa por se aproximar cada vez mais de soluções pertencentes
à fronteira de Pareto, e dadas limitações de memória e processamento que
inviabilizam a manutenção de todas as soluções candidatas já
descobertas, muitas propostas já foram feitas na literatura para se
definir que soluções candidatas já descobertas devem ser descartadas e
quais devem ser usadas como ponto de partida para a descoberta de novas
soluções. Como as metaheurísticas para MOO recorrem a muitos processos
de tomada de decisão durante a busca populacional, a ideia aqui proposta
consiste em utilizar técnicas de MCDM para ordenar por mérito as
soluções não-dominadas correspondentes no espaço dos objetivos, com o
intuito de obter novas soluções a partir daquelas que se mostram mais
promissoras. Soluções não-dominadas entre si geralmente são tomadas na
literatura como tendo a mesma relevância na definição da próxima
população de soluções candidatas (Deb et al., 2002). Espera-se, com
isso, chegar a técnicas de solução para MOO capazes de explorar ainda
mais eficazmente os recursos computacionais dispo- níveis, além de
potencialmente conduzir a soluções de melhor qualidade.

    \section{Proposta}\label{proposta}

A estratégia da técnica envolve a proposição de uma versão modificada de
um dos algoritmos considerados estado-da-arte em MOO, o NSGA-II (do
inglês \emph{Non-dominated Sorting Genetic Algorithm}) (Deb et al.,
2002). Esta metaheurística populacional realiza, a cada geração, uma
ordenação por não-dominância dos indivíduos. Os indivíduos são separados
em classes de dominância, e aqueles pertencentes às classes superiores
são selecionados para integrar a população da próxima geração. Para
comparar soluções candidatas pertencentes à mesma classe, utiliza-se uma
medida que indica qual solução deve ser escolhida para melhorar a
diversidade das soluções. No entanto, não é levada em conta nenhuma
informação adicional a respeito da preferência por determina- das
soluções neste processo. Neste trabalho, propomos adotar uma conhecida
técnica de tomada de decisão multicritério, o algoritmo TOPSIS (do
inglês \emph{Technique for Order of Preference by Similarity to Ideal
Solution}) (Hwang \& Yoon, 1981), para classificar soluções can- didatas
dentro de uma mesma classe de dominância segundo múltiplos critérios
estabelecidos por um tomador de decisão, como proximidade a valores de
referência e a própria promoção de diversidade da população.

    \section{Visualização dos
problemas-teste}\label{visualizauxe7uxe3o-dos-problemas-teste}

Este notebook apresenta um conjunto de problemas-teste que podem ser
utilizados para testar o método de otimização multiobjetivo. Trata-se de
um algoritmo evolutivo, que promove modificações iterativamente em uma
população de soluções candidatas. Ao final de sua execução, deseja-se
que os membros da população aproximem as soluções não-dominadas do
problema, que constituem uma região denominada Fronteira de Pareto.
Estas soluções apresentam o melhor compromisso possível entre os valores
das funções objetivo. As características da Fronteira de Pareto dependem
das funções objetivo de cada problema. Vamos apresentar aqui a
visualização da Fronteira de Pareto de alguns problemas-teste que o
algoritmo deve buscar aproximar.

    \subsection{Problemas ZDT}\label{problemas-zdt}

A classe de problemas-teste ZDT (Zitzler et al., 2000) apresenta 5
problemas (de ZDT1 a ZDT4 e ZDT6) com duas funções objetivo a serem
otimizadas. As fronteiras de Pareto dos problemas desta classe
apresentam diferentes características: convexas, não-convexas,
desconexas e não-uniformemente distribuídas. Os problemas possuem a
seguinte formulação:

\[
\min
\begin{cases}
f_1(\mathbf{x})\\
f_2(\mathbf{x}) = g(\mathbf{x}) \cdot h(f_1(\mathbf{x}),g(\mathbf{x}))
\end{cases}
\]

onde \(\mathbf{x}\) é o vetor de variáveis de decisão de uma solução
candidata, \(f_1(\mathbf{x}), f_2(\mathbf{x})\) são as funções objetivo
do problema a serem otimizadas, e \(g(\mathbf{x}), h(\mathbf{x})\) são
funções auxiliares. O número de variáveis de decisão e as funções variam
entre os 5 problemas da classe. A fronteira de Pareto é representada por
uma curva no espaço dos objetivos cujos pontos correspondem aos valores
de \(f_1(\mathbf{x})\) e \(f_2(\mathbf{x})\) que constituem as soluções
não dominadas do problema de otimização multiobjetivo. No caso dos
problemas ZDT, esta curva é dada pelos valores de \(\mathbf{x}\) tais
que \(g(\mathbf{x}) = 1\).

    O trecho de código a seguir apresenta uma função que salva em um arquivo
na pasta \emph{Pareto} um conjunto de \(N\) amostras da froteira de
Pareto do problema \emph{function}.

    \begin{Verbatim}[commandchars=\\\{\}]
{\color{incolor}In [{\color{incolor}1}]:} \PY{k+kn}{import} \PY{n+nn}{numpy} \PY{k+kn}{as} \PY{n+nn}{np}
        \PY{k+kn}{import} \PY{n+nn}{matplotlib.pyplot} \PY{k+kn}{as} \PY{n+nn}{plt}
        \PY{k+kn}{import} \PY{n+nn}{json}
        
        \PY{o}{\PYZpc{}}\PY{k}{matplotlib} inline
        
        \PY{k}{def} \PY{n+nf}{generateZDTPareto}\PY{p}{(}\PY{n}{function}\PY{p}{,}\PY{n}{N}\PY{p}{)}\PY{p}{:}
        
            \PY{n}{optimal\PYZus{}front} \PY{o}{=} \PY{n}{np}\PY{o}{.}\PY{n}{zeros}\PY{p}{(}\PY{p}{(}\PY{n}{N}\PY{p}{,}\PY{l+m+mi}{2}\PY{p}{)}\PY{p}{)}\PY{p}{;}
        
            \PY{n}{f1} \PY{o}{=} \PY{n}{np}\PY{o}{.}\PY{n}{linspace}\PY{p}{(}\PY{l+m+mi}{0}\PY{p}{,}\PY{l+m+mf}{1.0}\PY{p}{,}\PY{n}{N}\PY{p}{)}
        
            \PY{k}{if}\PY{p}{(}\PY{n}{function} \PY{o}{==} \PY{l+s+s1}{\PYZsq{}}\PY{l+s+s1}{ZDT1}\PY{l+s+s1}{\PYZsq{}} \PY{o+ow}{or} \PY{n}{function} \PY{o}{==} \PY{l+s+s1}{\PYZsq{}}\PY{l+s+s1}{ZDT4}\PY{l+s+s1}{\PYZsq{}}\PY{p}{)}\PY{p}{:}
        
                \PY{n}{f2} \PY{o}{=} \PY{l+m+mi}{1} \PY{o}{\PYZhy{}} \PY{n}{np}\PY{o}{.}\PY{n}{sqrt}\PY{p}{(}\PY{n}{f1}\PY{p}{)}
        
            \PY{k}{elif}\PY{p}{(}\PY{n}{function} \PY{o}{==} \PY{l+s+s1}{\PYZsq{}}\PY{l+s+s1}{ZDT2}\PY{l+s+s1}{\PYZsq{}} \PY{o+ow}{or} \PY{n}{function} \PY{o}{==} \PY{l+s+s1}{\PYZsq{}}\PY{l+s+s1}{ZDT6}\PY{l+s+s1}{\PYZsq{}}\PY{p}{)}\PY{p}{:}
        
                \PY{k}{if}\PY{p}{(}\PY{n}{function} \PY{o}{==} \PY{l+s+s1}{\PYZsq{}}\PY{l+s+s1}{ZDT6}\PY{l+s+s1}{\PYZsq{}}\PY{p}{)}\PY{p}{:}
        
                    \PY{n}{f1} \PY{o}{=} \PY{n}{np}\PY{o}{.}\PY{n}{linspace}\PY{p}{(}\PY{l+m+mf}{0.2807753191}\PY{p}{,}\PY{l+m+mf}{1.0}\PY{p}{,}\PY{n}{N}\PY{p}{)}
        
                \PY{n}{f2} \PY{o}{=} \PY{l+m+mi}{1} \PY{o}{\PYZhy{}} \PY{n}{f1}\PY{o}{*}\PY{o}{*}\PY{l+m+mi}{2}
        
            \PY{k}{elif}\PY{p}{(}\PY{n}{function} \PY{o}{==} \PY{l+s+s1}{\PYZsq{}}\PY{l+s+s1}{ZDT3}\PY{l+s+s1}{\PYZsq{}}\PY{p}{)}\PY{p}{:}
                \PY{n}{f1}\PY{p}{[}\PY{p}{:}\PY{n}{N}\PY{o}{/}\PY{l+m+mi}{5}\PY{p}{]} \PY{o}{=} \PY{n}{np}\PY{o}{.}\PY{n}{linspace}\PY{p}{(}\PY{l+m+mi}{0}\PY{p}{,}\PY{l+m+mf}{0.0830015349}\PY{p}{,}\PY{n}{N}\PY{o}{/}\PY{l+m+mi}{5}\PY{p}{)}
                \PY{n}{f1}\PY{p}{[}\PY{n}{N}\PY{o}{/}\PY{l+m+mi}{5}\PY{p}{:}\PY{l+m+mi}{2}\PY{o}{*}\PY{n}{N}\PY{o}{/}\PY{l+m+mi}{5}\PY{p}{]} \PY{o}{=} \PY{n}{np}\PY{o}{.}\PY{n}{linspace}\PY{p}{(}\PY{l+m+mf}{0.1822287280}\PY{p}{,}\PY{l+m+mf}{0.2577623634}\PY{p}{,}\PY{n}{N}\PY{o}{/}\PY{l+m+mi}{5}\PY{p}{)}
                \PY{n}{f1}\PY{p}{[}\PY{l+m+mi}{2}\PY{o}{*}\PY{n}{N}\PY{o}{/}\PY{l+m+mi}{5}\PY{p}{:}\PY{l+m+mi}{3}\PY{o}{*}\PY{n}{N}\PY{o}{/}\PY{l+m+mi}{5}\PY{p}{]} \PY{o}{=} \PY{n}{np}\PY{o}{.}\PY{n}{linspace}\PY{p}{(}\PY{l+m+mf}{0.4093136748}\PY{p}{,}\PY{l+m+mf}{0.4538821041}\PY{p}{,}\PY{n}{N}\PY{o}{/}\PY{l+m+mi}{5}\PY{p}{)}
                \PY{n}{f1}\PY{p}{[}\PY{l+m+mi}{3}\PY{o}{*}\PY{n}{N}\PY{o}{/}\PY{l+m+mi}{5}\PY{p}{:}\PY{l+m+mi}{4}\PY{o}{*}\PY{n}{N}\PY{o}{/}\PY{l+m+mi}{5}\PY{p}{]} \PY{o}{=} \PY{n}{np}\PY{o}{.}\PY{n}{linspace}\PY{p}{(}\PY{l+m+mf}{0.6183967944}\PY{p}{,}\PY{l+m+mf}{0.6525117038}\PY{p}{,}\PY{n}{N}\PY{o}{/}\PY{l+m+mi}{5}\PY{p}{)}
                \PY{n}{f1}\PY{p}{[}\PY{l+m+mi}{4}\PY{o}{*}\PY{n}{N}\PY{o}{/}\PY{l+m+mi}{5}\PY{p}{:}\PY{p}{]} \PY{o}{=} \PY{n}{np}\PY{o}{.}\PY{n}{linspace}\PY{p}{(}\PY{l+m+mf}{0.8233317983}\PY{p}{,}\PY{l+m+mf}{0.8518328654}\PY{p}{,}\PY{n}{N}\PY{o}{/}\PY{l+m+mi}{5}\PY{p}{)}
        
                \PY{n}{f2} \PY{o}{=} \PY{l+m+mi}{1} \PY{o}{\PYZhy{}} \PY{n}{np}\PY{o}{.}\PY{n}{sqrt}\PY{p}{(}\PY{n}{f1}\PY{p}{)} \PY{o}{\PYZhy{}}\PY{n}{f1}\PY{o}{*}\PY{n}{np}\PY{o}{.}\PY{n}{sin}\PY{p}{(}\PY{l+m+mi}{10}\PY{o}{*}\PY{n}{np}\PY{o}{.}\PY{n}{pi}\PY{o}{*}\PY{n}{f1}\PY{p}{)}
        
        
            \PY{n}{optimal\PYZus{}front}\PY{p}{[}\PY{p}{:}\PY{p}{,}\PY{l+m+mi}{0}\PY{p}{]} \PY{o}{=} \PY{n}{f1}
            \PY{n}{optimal\PYZus{}front}\PY{p}{[}\PY{p}{:}\PY{p}{,}\PY{l+m+mi}{1}\PY{p}{]} \PY{o}{=} \PY{n}{f2}
        
            \PY{n}{optimal\PYZus{}front} \PY{o}{=} \PY{n}{optimal\PYZus{}front}\PY{o}{.}\PY{n}{tolist}\PY{p}{(}\PY{p}{)}
            
            \PY{k}{with} \PY{n+nb}{open}\PY{p}{(}\PY{l+s+s1}{\PYZsq{}}\PY{l+s+s1}{\PYZsq{}}\PY{o}{.}\PY{n}{join}\PY{p}{(}\PY{p}{[}\PY{l+s+s1}{\PYZsq{}}\PY{l+s+s1}{../data/Prt\PYZus{}}\PY{l+s+s1}{\PYZsq{}}\PY{p}{,}\PY{n}{function}\PY{p}{,}\PY{l+s+s1}{\PYZsq{}}\PY{l+s+s1}{.json}\PY{l+s+s1}{\PYZsq{}}\PY{p}{]}\PY{p}{)}\PY{p}{,}\PY{l+s+s1}{\PYZsq{}}\PY{l+s+s1}{w}\PY{l+s+s1}{\PYZsq{}}\PY{p}{)} \PY{k}{as} \PY{n}{outfile}\PY{p}{:}
                \PY{n}{json}\PY{o}{.}\PY{n}{dump}\PY{p}{(}\PY{n}{optimal\PYZus{}front}\PY{p}{,}\PY{n}{outfile}\PY{p}{)}
\end{Verbatim}

    A função a seguir gera uma figura com a representação gráfica da
fronteira de Pareto do problema \emph{function}, fazendo uso do conjunto
de amostras que foi gerado e salvo em um arquivo previamente.

    \begin{Verbatim}[commandchars=\\\{\}]
{\color{incolor}In [{\color{incolor}2}]:} \PY{k}{def} \PY{n+nf}{plot\PYZus{}2dfunction}\PY{p}{(}\PY{n}{function}\PY{p}{)}\PY{p}{:}
            \PY{c+c1}{\PYZsh{}with open(\PYZsq{}\PYZsq{}.join([\PYZsq{}Pareto/Prt\PYZus{}\PYZsq{},function,\PYZsq{}.pk1\PYZsq{}]), \PYZsq{}r\PYZsq{}) as filename:}
                \PY{c+c1}{\PYZsh{}f = pickle.load(filename)}
             
            \PY{k}{with} \PY{n+nb}{open}\PY{p}{(}\PY{l+s+s1}{\PYZsq{}}\PY{l+s+s1}{\PYZsq{}}\PY{o}{.}\PY{n}{join}\PY{p}{(}\PY{p}{[}\PY{l+s+s1}{\PYZsq{}}\PY{l+s+s1}{../data/Prt\PYZus{}}\PY{l+s+s1}{\PYZsq{}}\PY{p}{,}\PY{n}{function}\PY{p}{,}\PY{l+s+s1}{\PYZsq{}}\PY{l+s+s1}{.json}\PY{l+s+s1}{\PYZsq{}}\PY{p}{]}\PY{p}{)}\PY{p}{)} \PY{k}{as} \PY{n}{optimal\PYZus{}front\PYZus{}data}\PY{p}{:}
                \PY{n}{f} \PY{o}{=} \PY{n}{np}\PY{o}{.}\PY{n}{array}\PY{p}{(}\PY{n}{json}\PY{o}{.}\PY{n}{load}\PY{p}{(}\PY{n}{optimal\PYZus{}front\PYZus{}data}\PY{p}{)}\PY{p}{)}
            \PY{k}{if}\PY{p}{(}\PY{n}{function} \PY{o}{==} \PY{l+s+s1}{\PYZsq{}}\PY{l+s+s1}{ZDT3}\PY{l+s+s1}{\PYZsq{}}\PY{p}{)}\PY{p}{:}
                \PY{n}{N} \PY{o}{=} \PY{n}{f}\PY{o}{.}\PY{n}{shape}\PY{p}{[}\PY{l+m+mi}{0}\PY{p}{]}
                \PY{n}{plt}\PY{o}{.}\PY{n}{plot}\PY{p}{(}\PY{n}{f}\PY{p}{[}\PY{p}{:}\PY{n}{N}\PY{o}{/}\PY{l+m+mi}{5}\PY{p}{,}\PY{l+m+mi}{0}\PY{p}{]}\PY{p}{,}\PY{n}{f}\PY{p}{[}\PY{p}{:}\PY{n}{N}\PY{o}{/}\PY{l+m+mi}{5}\PY{p}{,}\PY{l+m+mi}{1}\PY{p}{]}\PY{p}{,}\PY{n}{color}\PY{o}{=}\PY{l+s+s1}{\PYZsq{}}\PY{l+s+s1}{b}\PY{l+s+s1}{\PYZsq{}}\PY{p}{)}
                \PY{n}{plt}\PY{o}{.}\PY{n}{plot}\PY{p}{(}\PY{n}{f}\PY{p}{[}\PY{n}{N}\PY{o}{/}\PY{l+m+mi}{5}\PY{p}{:}\PY{l+m+mi}{2}\PY{o}{*}\PY{n}{N}\PY{o}{/}\PY{l+m+mi}{5}\PY{p}{,}\PY{l+m+mi}{0}\PY{p}{]}\PY{p}{,}\PY{n}{f}\PY{p}{[}\PY{n}{N}\PY{o}{/}\PY{l+m+mi}{5}\PY{p}{:}\PY{l+m+mi}{2}\PY{o}{*}\PY{n}{N}\PY{o}{/}\PY{l+m+mi}{5}\PY{p}{,}\PY{l+m+mi}{1}\PY{p}{]}\PY{p}{,}\PY{n}{color}\PY{o}{=}\PY{l+s+s1}{\PYZsq{}}\PY{l+s+s1}{b}\PY{l+s+s1}{\PYZsq{}}\PY{p}{)}
                \PY{n}{plt}\PY{o}{.}\PY{n}{plot}\PY{p}{(}\PY{n}{f}\PY{p}{[}\PY{l+m+mi}{2}\PY{o}{*}\PY{n}{N}\PY{o}{/}\PY{l+m+mi}{5}\PY{p}{:}\PY{l+m+mi}{3}\PY{o}{*}\PY{n}{N}\PY{o}{/}\PY{l+m+mi}{5}\PY{p}{,}\PY{l+m+mi}{0}\PY{p}{]}\PY{p}{,}\PY{n}{f}\PY{p}{[}\PY{l+m+mi}{2}\PY{o}{*}\PY{n}{N}\PY{o}{/}\PY{l+m+mi}{5}\PY{p}{:}\PY{l+m+mi}{3}\PY{o}{*}\PY{n}{N}\PY{o}{/}\PY{l+m+mi}{5}\PY{p}{,}\PY{l+m+mi}{1}\PY{p}{]}\PY{p}{,}\PY{n}{color}\PY{o}{=}\PY{l+s+s1}{\PYZsq{}}\PY{l+s+s1}{b}\PY{l+s+s1}{\PYZsq{}}\PY{p}{)}
                \PY{n}{plt}\PY{o}{.}\PY{n}{plot}\PY{p}{(}\PY{n}{f}\PY{p}{[}\PY{l+m+mi}{3}\PY{o}{*}\PY{n}{N}\PY{o}{/}\PY{l+m+mi}{5}\PY{p}{:}\PY{l+m+mi}{4}\PY{o}{*}\PY{n}{N}\PY{o}{/}\PY{l+m+mi}{5}\PY{p}{,}\PY{l+m+mi}{0}\PY{p}{]}\PY{p}{,}\PY{n}{f}\PY{p}{[}\PY{l+m+mi}{3}\PY{o}{*}\PY{n}{N}\PY{o}{/}\PY{l+m+mi}{5}\PY{p}{:}\PY{l+m+mi}{4}\PY{o}{*}\PY{n}{N}\PY{o}{/}\PY{l+m+mi}{5}\PY{p}{,}\PY{l+m+mi}{1}\PY{p}{]}\PY{p}{,}\PY{n}{color}\PY{o}{=}\PY{l+s+s1}{\PYZsq{}}\PY{l+s+s1}{b}\PY{l+s+s1}{\PYZsq{}}\PY{p}{)}
                \PY{n}{plt}\PY{o}{.}\PY{n}{plot}\PY{p}{(}\PY{n}{f}\PY{p}{[}\PY{l+m+mi}{4}\PY{o}{*}\PY{n}{N}\PY{o}{/}\PY{l+m+mi}{5}\PY{p}{:}\PY{p}{,}\PY{l+m+mi}{0}\PY{p}{]}\PY{p}{,}\PY{n}{f}\PY{p}{[}\PY{l+m+mi}{4}\PY{o}{*}\PY{n}{N}\PY{o}{/}\PY{l+m+mi}{5}\PY{p}{:}\PY{p}{,}\PY{l+m+mi}{1}\PY{p}{]}\PY{p}{,}\PY{n}{color}\PY{o}{=}\PY{l+s+s1}{\PYZsq{}}\PY{l+s+s1}{b}\PY{l+s+s1}{\PYZsq{}}\PY{p}{)}
            \PY{k}{else}\PY{p}{:}
                \PY{n}{plt}\PY{o}{.}\PY{n}{plot}\PY{p}{(}\PY{n}{f}\PY{p}{[}\PY{p}{:}\PY{p}{,}\PY{l+m+mi}{0}\PY{p}{]}\PY{p}{,}\PY{n}{f}\PY{p}{[}\PY{p}{:}\PY{p}{,}\PY{l+m+mi}{1}\PY{p}{]}\PY{p}{,}\PY{n}{color}\PY{o}{=}\PY{l+s+s1}{\PYZsq{}}\PY{l+s+s1}{b}\PY{l+s+s1}{\PYZsq{}}\PY{p}{)}
            \PY{n}{plt}\PY{o}{.}\PY{n}{xlabel}\PY{p}{(}\PY{l+s+s1}{\PYZsq{}}\PY{l+s+s1}{f1}\PY{l+s+s1}{\PYZsq{}}\PY{p}{)}
            \PY{n}{plt}\PY{o}{.}\PY{n}{ylabel}\PY{p}{(}\PY{l+s+s1}{\PYZsq{}}\PY{l+s+s1}{f2}\PY{l+s+s1}{\PYZsq{}}\PY{p}{)}
            \PY{n}{plt}\PY{o}{.}\PY{n}{show}\PY{p}{(}\PY{p}{)}
\end{Verbatim}

    \subsubsection{ZDT1}\label{zdt1}

O problema ZDT1 possui \(n=30\) variáveis definidas no domínio
\(\mathbf{x}_i \in [0;1], i=1,...,n\) e apresenta fronteira de Pareto
contínua, convexa e uniformemente distribuída. Suas funções estão
definidas a seguir:

\[ 
\begin{cases}
f_1(\mathbf{x}) = \mathbf{x}_1\\
g(\mathbf{x}) = 1 + \frac{9}{n-1} \sum \limits_{i=2}^{n} \mathbf{x}_i\\
h(f_1(\mathbf{x}),g(\mathbf{x})) = 1 - \sqrt[]{\frac{f_1(\mathbf{x})}{g(\mathbf{x})}}
\end{cases}
\]

A fronteira de Pareto do problema pode ser vista na figura abaixo.

    \begin{Verbatim}[commandchars=\\\{\}]
{\color{incolor}In [{\color{incolor}3}]:} \PY{n}{function} \PY{o}{=} \PY{l+s+s1}{\PYZsq{}}\PY{l+s+s1}{ZDT1}\PY{l+s+s1}{\PYZsq{}}
        \PY{n}{N} \PY{o}{=} \PY{l+m+mi}{1000}
        \PY{n}{generateZDTPareto}\PY{p}{(}\PY{n}{function}\PY{p}{,}\PY{n}{N}\PY{p}{)}
        \PY{n}{plot\PYZus{}2dfunction}\PY{p}{(}\PY{n}{function}\PY{p}{)}
\end{Verbatim}

    \begin{center}
    \adjustimage{max size={0.9\linewidth}{0.9\paperheight}}{output_9_0.png}
    \end{center}
    { \hspace*{\fill} \\}
    
    \subsubsection{ZDT2}\label{zdt2}

O problema ZDT2 possui \(n=30\) variáveis definidas no domínio
\(\mathbf{x}_i \in [0;1], i=1,...,n\) e apresenta fronteira de Pareto
contínua, não-convexa e uniformemente distribuída. Suas funções estão
definidas a seguir:

\[ 
\begin{cases}
f_1(\mathbf{x}) = \mathbf{x}_1\\
g(\mathbf{x}) = 1 + \frac{9}{n-1} \sum \limits_{i=2}^{n} \mathbf{x}_i\\
h(f_1(\mathbf{x}),g(\mathbf{x})) = 1 - \left(\frac{f_1(\mathbf{x})}{g(\mathbf{x})}\right)^2
\end{cases}
\]

A fronteira de Pareto do problema pode ser vista na figura abaixo.

    \begin{Verbatim}[commandchars=\\\{\}]
{\color{incolor}In [{\color{incolor}4}]:} \PY{n}{function} \PY{o}{=} \PY{l+s+s1}{\PYZsq{}}\PY{l+s+s1}{ZDT2}\PY{l+s+s1}{\PYZsq{}}
        \PY{n}{N} \PY{o}{=} \PY{l+m+mi}{1000}
        \PY{n}{generateZDTPareto}\PY{p}{(}\PY{n}{function}\PY{p}{,}\PY{n}{N}\PY{p}{)}
        \PY{n}{plot\PYZus{}2dfunction}\PY{p}{(}\PY{n}{function}\PY{p}{)}
\end{Verbatim}

    \begin{center}
    \adjustimage{max size={0.9\linewidth}{0.9\paperheight}}{output_11_0.png}
    \end{center}
    { \hspace*{\fill} \\}
    
    \subsubsection{ZDT3}\label{zdt3}

O problema ZDT3 possui \(n=30\) variáveis definidas no domínio
\(\mathbf{x}_i \in [0;1], i=1,...,n\) e apresenta fronteira de Pareto
não-contínua. Suas funções estão definidas a seguir:

\[ 
\begin{cases}
f_1(\mathbf{x}) = \mathbf{x}_1\\
g(\mathbf{x}) = 1 + \frac{9}{n-1} \sum \limits_{i=2}^{n} \mathbf{x}_i\\
h(f_1(\mathbf{x}),g(\mathbf{x})) = 1 - \sqrt[]{\frac{f_1(\mathbf{x})}{g(\mathbf{x})}} - \left(\frac{f_1(\mathbf{x})}{g(\mathbf{x})}\right) \cdot \sin (10 \cdot \pi \cdot f_1(\mathbf{x}))
\end{cases}
\]

A fronteira de Pareto do problema pode ser vista na figura abaixo.

    \begin{Verbatim}[commandchars=\\\{\}]
{\color{incolor}In [{\color{incolor}5}]:} \PY{n}{function} \PY{o}{=} \PY{l+s+s1}{\PYZsq{}}\PY{l+s+s1}{ZDT3}\PY{l+s+s1}{\PYZsq{}}
        \PY{n}{N} \PY{o}{=} \PY{l+m+mi}{1000}
        \PY{n}{generateZDTPareto}\PY{p}{(}\PY{n}{function}\PY{p}{,}\PY{n}{N}\PY{p}{)}
        \PY{n}{plot\PYZus{}2dfunction}\PY{p}{(}\PY{n}{function}\PY{p}{)}
\end{Verbatim}

    \begin{center}
    \adjustimage{max size={0.9\linewidth}{0.9\paperheight}}{output_13_0.png}
    \end{center}
    { \hspace*{\fill} \\}
    
    \subsubsection{ZDT4}\label{zdt4}

O problema ZDT4 possui \(n=10\) variáveis definidas no domínio
\(\mathbf{x}_1 \in [0;1]\) e \(\mathbf{x}_i \in [-5;5], i=2,...,n\) e
apresenta fronteira de Pareto contínua e convexa. O problema apresenta
diversas soluções Pareto-ótimas locais, que podem dificultar a
aproximação da fronteira de Pareto real. Suas funções estão definidas a
seguir:

\[ 
\begin{cases}
f_1(\mathbf{x}) = \mathbf{x}_1\\
g(\mathbf{x}) = 1 + 10(n-1) + \sum \limits_{i=2}^{n} [\mathbf{x}_i^2 - 10 \cdot \cos (4 \cdot \pi \cdot \mathbf{x}_i)]\\
h(f_1(\mathbf{x}),g(\mathbf{x})) = 1 - \sqrt[]{\frac{f_1(\mathbf{x})}{g(\mathbf{x})}}
\end{cases}
\]

A fronteira de Pareto do problema pode ser vista na figura abaixo.

    \begin{Verbatim}[commandchars=\\\{\}]
{\color{incolor}In [{\color{incolor}6}]:} \PY{n}{function} \PY{o}{=} \PY{l+s+s1}{\PYZsq{}}\PY{l+s+s1}{ZDT4}\PY{l+s+s1}{\PYZsq{}}
        \PY{n}{N} \PY{o}{=} \PY{l+m+mi}{1000}
        \PY{n}{generateZDTPareto}\PY{p}{(}\PY{n}{function}\PY{p}{,}\PY{n}{N}\PY{p}{)}
        \PY{n}{plot\PYZus{}2dfunction}\PY{p}{(}\PY{n}{function}\PY{p}{)}
\end{Verbatim}

    \begin{center}
    \adjustimage{max size={0.9\linewidth}{0.9\paperheight}}{output_15_0.png}
    \end{center}
    { \hspace*{\fill} \\}
    
    \subsubsection{ZDT6}\label{zdt6}

O problema ZDT6 possui \(n=10\) variáveis definidas no domínio
\(\mathbf{x}_i \in [0;1], i=1,...,n\) e apresenta fronteira de Pareto
contínua, não-convexa e não uniformemente distribuída. Suas funções
estão definidas a seguir:

\[ 
\begin{cases}
f_1(\mathbf{x}) = 1 - \exp (-4\mathbf{x}_1) \cdot \sin ^6 (6 \cdot \pi \cdot \mathbf{x}_1) \\
g(\mathbf{x}) = 1 + 9 \cdot \left(\frac{\sum \limits_{i=2}^{n} \mathbf{x}_i}{9}\right)^{0.25} \\
h(f_1(\mathbf{x}),g(\mathbf{x})) = 1 - \left(\frac{f_1(\mathbf{x})}{g(\mathbf{x})}\right)^2
\end{cases}
\]

A fronteira de Pareto do problema pode ser vista na figura abaixo.

    \begin{Verbatim}[commandchars=\\\{\}]
{\color{incolor}In [{\color{incolor}7}]:} \PY{n}{function} \PY{o}{=} \PY{l+s+s1}{\PYZsq{}}\PY{l+s+s1}{ZDT6}\PY{l+s+s1}{\PYZsq{}}
        \PY{n}{N} \PY{o}{=} \PY{l+m+mi}{1000}
        \PY{n}{generateZDTPareto}\PY{p}{(}\PY{n}{function}\PY{p}{,}\PY{n}{N}\PY{p}{)}
        \PY{n}{plot\PYZus{}2dfunction}\PY{p}{(}\PY{n}{function}\PY{p}{)}
\end{Verbatim}

    \begin{center}
    \adjustimage{max size={0.9\linewidth}{0.9\paperheight}}{output_17_0.png}
    \end{center}
    { \hspace*{\fill} \\}
    
    \subsection{Referências}\label{referuxeancias}

Branke, J.; DEB, K.; Miettinen, K.; Slowinski, R. Multiobjective
Optimization: Interactive and Evolutionary Approaches. Springer, 2008.

Coello Coello, C., Lamont, G. B. \& Van Veldhuizen, D. A., 2007.
Evolutionary Algorithms for Solving Multi-Objective Problems.
s.l.:Springer.

Deb, K., Pratap, A., Agarwal, S. \& Meyarivan, T., 2002. A fast and
elitist multiobjective genetic algorithm: NSGA-II. IEEE Transactions on
Evolutionary Computation, 6(2), pp. 182-197.

Hwang, C. \& Yoon, K., 1981. Multiple Attributes Decision Making Methods
and Applications. Berlin: Springer-Verlag.

Köksalan, M., Wallenius, J. \& Zionts, S., 2011. Multiple Criteria
Decision Making: From Early History to the 21st Century. s.l.:World
Scientific.

Zitzler, E.; Deb, K.; Thiele, L. Comparison of multiobjective
evolutionary algorithms: Empirical results. Evolutionary Computation, v.
8, p. 173--195, 2000.


    % Add a bibliography block to the postdoc
    
    
    
    \end{document}
